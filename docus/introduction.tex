\selectlanguage{american}%

Estimation of abundance and of other demographic parameters such as
survival is a key part of wildlife management and conservation. Traditional
mark-recapture analysis \citep{williams_analysis_2002} can deliver
estimates with low bias and uncertainty, provided that enough individual
animals i) are naturally, artificially, or genetically “marked” and
identifiable and ii) can be recaptured over time. If genotypes are
used as the marks, as in genetic individual mark-recapture (IMR; \citealp{palsboll_genetic_1997}),
then kinship patterns amongst the samples (parents, siblings, etc)
contains additional information relevant to demographics \citep{skaug_allele-sharing_2001}.
Close-kin mark-recapture (CKMR; see \citealp{bravington_close-kin_2016})
is a framework for using these kinships, as inferred from genotypes,
to estimate abundance and demographic parameters. CKMR provides additional
flexibility compared with IMR since lethal samples (from sampling,
hunting, natural mortality etc.) and/or non-lethal samples can be
used; it also increases the effective sample size, since more types
of ``recapture'' are possible. As of 2025, most CKMR projects have
been for commercial fish (e.g., \citealp{Davies2020SBT2}) or sharks
(e.g., \citealp{Hillary2018WS-CKMR}), but there are also some for
mammals, including \citet{conn_robustness_2020}'s modeling study
of bearded seals and its implementation by \citet{taras_estimating_2024},
and \citet{lloyd-jones_close-kin_2023} for flying foxes.

The principle behind CKMR is that every individual has (or had) one
mother and one father; thus, for a given sample size, in a large population
there will be few ``recaptures'' of parents or their other descendants,
while in a small population there will be many. In practice, the data
for CKMR comprise the outcome of pairwise kinship checks amongst samples,
plus covariates associated with each sample such as its date of capture,
age, size, sex etc. The CKMR model has two components: a population-dynamics
part driven by the demographic parameters; and formulae for the expected
frequencies of different kinship types in pairwise comparisons, conditional
on sample covariates and population dynamics. By combining the kinship
data with the model, parameters can be estimated using maximum-likelihood
or Bayesian methods.

CKMR has mostly been used in situations where self-recaptures are
unlikely or impossible (e.g., because sampling is lethal). \citet{lloyd-jones_close-kin_2023}
did include IMR results in a CKMR study but did not integrate both
datasets into a single model. Here, we focus on a population where
IMR was the original project goal; therefore we extend traditional
CKMR to include IMR in the same model as an additional kinship type,
whereby pairwise genetic comparison can show that two samples are
from the same animal.

The success of CKMR and/or IMR depends on whether data collected contain
sufficient recaptures. Sampling design (e.g. number of samples, composition,
study duration, quality of covariate measurements) is crucial to avoid
expensive, embarrassing, and predictable failure. The pairwise-comparison
framework leads to analytical results for expected number of kin-pairs
and expected variance given expected number of samples (and associated
covariates), so that simulation is not essential; nevertheless, simulation
can sometimes be useful as a way to check the fairly complex code
of kinship probabilities and design setup. In this paper we show how
to do and check the calculations using a case study on the Pacific
walrus (\emph{Odobenus rosmarus divergens}; hereafter, walrus) in
the North Pacific. We explore different demographic and design scenarios
for walrus using IMR alone versus CKMR + IMR = ICKMR, and demonstrate
how the latter can be used to substantially reduce the overall amount
of survey effort required for adequate monitoring.

In the rest of this Introduction, we provide some background on CKMR
(drawn from \citealt{bravington_close-kin_2016} and experience on
numerous projects since), and on walrus biology and the survey setup.
In Methods, we describe our walrus population dynamics model, derive
walrus-appropriate kinship probability formulae, and show how to analytically
calculate the expected variances that might come from different survey
designs. We also outline the simulation setup which we used to test
our CKMR model. The Results section shows how different survey designs
are likely to perform (e.g., with\slash without CKMR). In the Discussion,
we summarize our conclusions for walrus, and also mention some modeling
simplifications made for design purposes that we may wish to reconsider
when working with real data.

\subsection{Close-kin practicalities}

Most iteroparous species are potential candidates for CKMR, except
for a few pathological cases such as those mentioned in \citet{bravington_close-kin_2016}.
The data requirements are otherwise quite flexible; the kinships used
in any particular study can vary depending on logistic and modeling
considerations. Genotyping methods are beyond the scope of this paper
but, given good-quality tissue samples, modern high-throughput sequencing
and microarrays are cheap and reliable enough to permit reliable and
(almost) unambiguous detection of small numbers of kin-pairs within
large sample sizes {[}REF NEEDED{]}. With current genetic methods,
the the informative and usable types of ``close'' kin are usually
limited to: Parent-Offspring Pairs (POPs) and Half-Sibling Pairs (HSPs,
which share one parent). In some applications, Full-Sibling Pairs
(FSP), and other ``second-order kin'' beside HSPs such as Grandparent-Grandchild
Pairs (GGPs), may also be important. To deal with IMR in the same
framework, we add the kinship category Self-Pair (SP). Any other kinship
constitutes an ``Unrelated Pair'' (UP) for modelling purposes, even
if there is genetic evidence of weak relatedness.

Some information about sample age is required. Often this comes from
``hard parts'' such as teeth or otoliths (requiring lethal samples),
or less accurately from covariates such as body size or from a visual
assessment of life-stage. The advent of epigenetic ageing (e.g. \citealp{Weber2024DNAgeray,Robeck2023DNAgepinni,Peters2023DNAgebotdol})
makes it possible to use just the biopsy sample itself. Uncertainty
in age and other covariate estimates can be accommodated within the
CKMR model, though the precision of parameter estimates is of course
linked to the accuracy of covariate measurements.

\selectlanguage{english}%
The basic principle behind kinship probabilities is ``ERRO'' (Expected
Relative Reproductive Output). Roughly: the chance of any particular
adult sample being the parent of some offspring that was sampled independently,
is the ratio of that adult's expected fecundity to the total fecundity
of all parents at the time the offspring was born. \foreignlanguage{american}{Devising
the formulae for the expected frequencies of kinships requires at
least qualitative consideration to the life history and reproductive
biology of the species, as well as aspects of the sampling. For example:
does the population breed in one place and at one time? Does the species
have a polygynous mating structure? Are breeding animals as likely
to be sampled as ``resting'' ones? Not everything matters, but anything
that affects ERRO should, in principle, be allowed for in the model.
We give walrus-specific examples in Methods.}
\selectlanguage{american}%

\subsection{Walrus biology and background}

The walrus is a gregarious, ice-associated pinniped inhabiting continental
shelf waters of the Bering and Chukchi seas. During winter (when sea
ice forms south of the Bering Strait) virtually all walruses occupy
the Bering Sea \citep{fay_ecology_1982}. In summer (when sea ice
is absent from the Bering Sea) almost all juvenile and adult female
walruses, and some adult male walruses, migrate north to the Chukchi
Sea. When walruses rest offshore on sea ice floes, their distribution
is dynamic, because it generally follows the marginal ice zone (a
moving, changing habitat which contains a mix of ice floes and water)
but also concentrates in regions of high benthic productivity. This
allows walruses to forage for benthic invertebrates while simultaneously
having access to a nearby substrate for hauling out.

{*}{*}{*}NEED something about walrus moving about all over the place,
from IMR data and (more likely) sat tags :) Some of that {*}could{*}
go to the Discussion, but I think at least a pre-mention here, coz
it will otherwise be in the alert reader's mind as they look at the
model structure

{*}{*}{*}Walrus reprod biol summary should go here? Rather than putting
it off until Methods.

Sea ice has declined for decades \citep{perovich_loss_2009,stroeve_trends_2012,stroeve_changing_2018},
and coupled global atmospheric-ocean general circulation models predict
its continued decline \citep{arthun_seasonal_2021}. When sea ice
recedes from the continental shelf, walruses come on shore to rest
in large herds at sites termed haulouts, from which they make long
trips to foraging hotspots \citep{jay_walrus_2012}. This change in
their activity budgets \citep{jay_walrus_2017} may ultimately lead
to a decline in body condition and an increase in mortality or a decrease
in reproduction \citep{udevitz_forecasting_2017}. Furthermore, disturbance
at haulouts can cause stampedes, resulting in mass calf and juvenile
mortality. Continued sea-ice loss and a concomitant increase in the
intensity and expansion of industrial and shipping activities in Pacific
Arctic waters \citep{silber_vessel_2019} are expected to drive a
substantial population decline \citep{garlich-miller_status_2011,maccracken_final_2017,johnson_assessing_2023,johnson_assessing_2024}.

Range-wide abundance and demographic rate estimates are crucial for
understanding population status, as well as for developing and implementing
harvest management plans. In particular, subsistence walrus harvests
in Alaska and Chukotka exceed 4,000 animals annually (USFWS, 2023),
and indigenous peoples need information on the status of the walrus
population in order to manage these harvests sustainably. Furthermore,
in the United States, the Marine Mammal Protection Act (MMPA) requires
a determination of potential biological removal for walrus, which
in turn, requires a precise abundance estimate \citep{gilbert_review_1999,wade_determining_1999}.

Scientists have attempted to ascertain walrus population size since
at least 1880 \citep{fay_managing_1989}, and until very recently,
unsuccessfully. The most concerted effort was the 1975-2006 range-wide
airplane-based surveys conducted collaboratively with the Soviet Union
and then Russian Federation. However, resulting estimates were biased
and imprecise, and count-based methods were abandoned after the 2006
survey which, despite a rigorous design, innovative field methods,
and sophisticated analyses, yielded a 95\% confidence interval (CI)
on the population size estimate of 55,000–507,000 animals (CV = 0.93).
The extensive imprecision in the estimate resulted from the walrus
population being widely dispersed with unpredictable local clumping
\citep{speckman_results_2011,jay_walrus_2012}, which is, in turn,
due to the large area of arctic and subarctic continental shelf over
which they forage, their gregarious nature, and the dynamic nature
of the marginal ice zone.

The first rigorous walrus survival rate estimates were obtained within
the past decade via Bayesian integrated population models (IPMs),
which combined multiple data sources to estimate demographic rates
and population trend over multiple decades \citep{taylor_demography_2015,taylor_demography_2018}.
However, the original problems with the aerial survey data continued
to preclude conclusions about population abundance in the IPMs \citep{taylor_demography_2015}.

In 2013, the U.S. Fish and Wildlife Service (FWS) initiated a genetic
IMR project to estimate walrus abundance and demographic rates. Under
this approach, genetic ``marking'' via skin biopsy samples \citep{palsboll_genetic_1997}
provided a major advantage over traditional marking techniques because
walruses are extremely difficult to handle physically. Over five years
of research cruises, biologists attempted to collect a representative
sample of walruses in the accessible portion of the marginal ice zone
in each year a cruise was conducted, although Russian waters were
not accessible in all years. Sampling focused on groups of adult females
and juveniles, as these classes are the demographically important
population segments of this polygynous species \citep{fay_ecology_1982}.
Further methods for the IMR study are detailed by Beatty et al. \citet{beatty_panmixia_2020}
and \citet{beatty_estimating_2022}.

Data analysis from the first generation of walrus research cruises
(2013–2017) used a Cormack-Jolly-Seber multievent model to estimate
survival rates, and a Horvitz-Thompson-like estimator to obtain population
size. The total abundance of 257,000 had a 95\% credible interval
(CrI) of 171,000–366,000 (CV=0.19; \citealt{beatty_estimating_2022}).
Although the precision of the abundance estimate from the IMR study
was much improved over the final aerial survey, the IMR study required
extensive investment of human and financial resources (i.e, USD \$5,000,000).
A more cost-effective approach is needed to assess the walrus population
on a regular interval. As mentioned above, biopsy samples also contain
information about kin relationships, which, through CKMR, can substantially
augment the information content of genetic IMR without increasing
sampling effort. {[}2030 words{]}.\selectlanguage{english}%

