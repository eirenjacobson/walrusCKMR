\setcounter{section}{0}
\renewcommand{\thesection}{\Alph{section}}
\numberwithin{equation}{section}

\section{Derivation of self-recapture ``the other way round''}

As discussed in Section \ref{subsec:selfPs}, \eqref{eq:self-staged}
can also be formulated \textquotedbl the other way round\textquotedbl ,
i.e., considering whether the second sample is the same as the first.
The answer turns out the same, but the derivation is slightly different
and \emph{appears} to involve an explicit survival term. Again, suppose
two female samples ($y_{1},a_{1}$ and $y_{2},a_{2}$ , where $y_{1}<y_{2}$),
then 
\begin{gather*}
\mathbb{P}\left[K_{21}=\text{SP}\vert y_{1},a_{1},y_{2},a_{2}\right]\\
=\frac{\mathbb{P}\left[\text{Sample 1 survived until Sample 2 was taken}\right]\mathbb{I}\left(y_{2}-a_{2}=y_{1}-a_{1}\right)}{N\left(y_{2},a_{2}\right)}\\
=\frac{\Phi\left(y_{2}-y_{1},a_{1}\right)\mathbb{I}\left(y_{2}-a_{2}=y_{1}-a_{1}\right)}{N\left(y_{2},a_{2}\right)}.
\end{gather*}

However, the results are readily seen to be identical because, by
definition of \textquotedbl survival\textquotedbl , we have
\begin{gather}
N\left(y+t,a+t\right)\equiv N\left(y,a\right)\Phi\left(t,a\right).\label{eq:cons-of-nums}
\end{gather}

\pagebreak{}

\section{Self-recapture when exact age is known\label{subsec:selfP-exact-age}}

\citet{beatty_estimating_2022} used a fairly complex IMR formulation
to cope with historically-very-imprecise estimates of age (or, more
realistically, of \textquotedbl stage\textquotedbl ) estimates.
However, when accurate age data are available, the pairwise comparison
probabilities for self-recapture are remarkably simple. Suppose two
female samples ($y_{1},a_{1}$ and $y_{2},a_{2}$ , where $y_{1}<y_{2}$).
Then the probability that the first one is the same as the second
is just
\begin{gather}
\mathbb{P}\left[K_{12}=\text{SP}\vert y_{1},a_{1},y_{2},a_{2}\right]=\frac{\mathbb{I}\left(y_{2}-a_{2}=y_{1}-a_{1}\right)}{N_{y_{1},a_{1}}}.\label{eq:SP}
\end{gather}
The indicator $\mathbb{I}\left(\cdot\right)$ is 1 if the two samples
were born in the same year, or 0 if not, The samples can only be from
the same animal if they were both born in the same year and if they
were, we then need to know how many females of age $a_{1}$ were alive
at $y_{1}$, $N_{y_{1},a_{1}}$. This implicitly assumes that all
females of the same age have the same survival and sampling probabilities.
(See appendix for the equivalent derivation of $\mathbb{P}\left[K_{21}=\text{SP}\vert y_{1},a_{1},y_{2},a_{2}\right]$).

In principle, given unlimited data, we could separately apply \eqref{eq:SP}
to each combination of $\left(y,a\right)$-consistent pairs, to empirically
estimate from all numbers-at-age-and-year from the reciprocal of the
observed rates. Then we could apply \eqref{eq:cons-of-nums} to estimate
year-and-age-specific survivals. In practice, that would be ridiculous,
since it would require an enormous number of recaptures and would
lead to noisy abundance estimates, estimated survivals greater than
one, and so on. However, the principle does illustrate the great power
of \emph{known-age} mark-recapture data. Note also that there are
no assumptions about equiprobable sampling across ages, etc; all probabilities
are simply conditioned on observed ages, and it does not particularly
matter \emph{why} there are more samples of one age than another.

The big problem with applying \eqref{eq:SP} in an ICKMR setting,
i.e., with conditioning on age explicitly, is that it requires explicit
calculation of all $N_{y_{1},a_{1}}$ within the model. This is normally
unnecessary with CKMR for mammal-like species, where the main information
is \emph{only} connected with aggregate adult abundance (via TRO).
It is extremely convenient to work just with a \textquotedbl homogenous
block\textquotedbl{} of adults, and there is in any case no direct
information on population age composition unless extra data are used.
One option is \textquotedbl just\textquotedbl{} to work with a fully-age-structured
population dynamics framework— but that is a lot of work to develop
(from experience in fisheries work) and requires modelling extra data.\pagebreak{}

\section{Derivation of juvenile abundance\label{sec:Deriv-Njy}}

The key point here is that we don't need to decompose the adult stage
into separate age classes.

Following notation from the rest of the paper, let the number of adults
in year $y$ be $N_{\text{A},t}$ where adulthood means being aged
$\alpha$ or older. The number next year will be $\rho N_{\text{A},y+1}$
where $\rho=e^{r}$ and $r$ is the rate of increase as in \eqref{popdyn}.
That will be made up of survivors from adults at $t$, plus survivors
from the incoming cohort of oldest juveniles, aged $\alpha-1$. Thus
\begin{gather}
N_{y+1,\text{A}}=\rho N_{y,\text{A}}=\phi_{\text{A}}N_{y,\text{A}}+\phi_{\text{J}}N_{y,\alpha-1}.\label{eq:mvb-nj-1}
\end{gather}
Rearranging, we have
\begin{gather}
N_{y,\alpha-1}=\frac{\rho-\phi_{\text{A}}}{\phi_{\text{J}}}N_{y,\text{A}}.\label{eq:mvb-nj-final-juve}
\end{gather}
We now need to infer the numbers in the other juvenile age-classes
(not just $\alpha-1$). Starting with the penultimate juvenile age-class,
we have: 
\begin{align*}
N_{y,\alpha-1} & =\phi_{\text{J}}N_{y-1,\alpha-2} & \text{ (survival)}\\
N_{y,\alpha-1} & =\rho N_{y-1,\alpha-1} & \text{ (population growth)}\\
\implies N_{y,\alpha-2} & =\frac{\rho}{\phi_{\text{J}}}N_{y,\alpha-1}.
\end{align*}
Similar relationships apply to each preceding juvenile age class,
down to age 1. The total number of juveniles in year $y$, $N_{y,\text{J}}$,
is given by a sum from age $x=\alpha-1$ down to age 1:
\begin{align}
N_{y,\text{J}}=\sum_{x=1}^{\alpha-1}N_{y,\alpha-x} & =\sum_{x=1}^{\alpha-1}N_{y,\alpha-1}\left(\frac{\rho}{\phi_{\text{J}}}\right)^{x-1}\nonumber \\
 & =N_{y,\alpha-1}\sum_{x'=0}^{\alpha-2}\left(\frac{\rho}{\phi_{\text{J}}}\right)^{x'}\nonumber \\
 & =N_{y,\alpha-1}\frac{1-\left(\rho/\phi_{\text{J}}\right)^{\alpha-1}}{1-\rho/\phi_{\text{J}}},\label{eq:mvb-nj-totjuve}
\end{align}
using the standard result for a geometric series: $\sum_{i=1}^{n}ar^{i}=a\frac{1-r^{n}}{1-r}$.
Substituting for $N_{t,\alpha-1}$ from \eqref{eq:mvb-nj-final-juve},
we have
\begin{align*}
N_{y,\text{J}} & =N_{y,\text{A}}\frac{\rho-\phi_{\text{A}}}{\phi_{\text{J}}}\frac{1-\left(\frac{\rho}{\phi_{\text{J}}}\right)^{\alpha-1}}{1-\frac{\rho}{\phi_{\text{J}}}}\\
 & =N_{y,\text{A}}\frac{\rho-\phi_{\text{A}}}{\rho-\phi_{\text{J}}}\left(\left(\frac{\rho}{\phi_{\text{J}}}\right)^{\alpha-1}-1\right).
\end{align*}
Now, for the case of walrus, we know that $\alpha=6$, so:
\begin{align*}
N_{y,\text{J}} & =N_{y,\text{A}}\frac{\rho-\phi_{\text{A}}}{\rho-\phi_{\text{J}}}\left(\left(\frac{\rho}{\phi_{\text{J}}}\right)^{5}-1\right).
\end{align*}
\pagebreak{}

\section{Further HSP complications}

The second issue with all second-order kin, is that pairwise-kinship
statistics are not currently powerful enough to completely distinguish
them all from a few ``lucky'' third-order kin such as Great-Gandparent-Grandchild.
To handle this without bias, the best approach is set a threshold
for the statistic that should almost completely exclude false-positives
from third-order kin, then to estimate empirically the proportion
of true second-order kin that will be lost below the threshold (i.e.,
the false-negative rate) based on the observed distribution of kin-pair
statistics. Only kin-pairs that are above the threshold will be treated
as HSPs, but the probability formula can be multiplied by the complement
of the false-negative probability to compensate. See \citet{bravington_close-kin_2016}
or \citet{Hillary2018WS-CKMR} for more details. The false-negative
rate depends both on the species and the genotyping method (in particular,
the number of loci) and cannot be predicted in advance, but experience
suggests that 15\% is usually a safe upper limit.

Determining that a pair is HSP does not differentiate between mHSPs
(maternal; shared mother) and pHSPs (paternal; shared father). This
can be determined by genotyping the mitochondrial DNA (mtDNA; always
inherited from the mother only) of known HSPs. If the genotypes are
different, the descent must be paternal; if the same, descent is probably
maternal, but could arise by chance in a few paternal-HSP cases. However,
in our experience, except for very small populations (hundreds of
adults), mtDNA diversity has always been high enough that shared-mtDNA
HSPs might as well be treated as definite mHSPs. We assume as much
here.

\pagebreak{}

\section{Adjustments for non-sparse sampling\label{appendix:Adjustments-for-non-sparse}}

Use of the pseudo-log-likelihood Hessian to approximate the inverse
variance is not strictly justified in a mathematical sense, because
the pairwise comparisons are not fully mutually independent. The ``sparse
sampling'' assumption of \citet{bravington_close-kin_2016}, which
underlies the use of the Hessian, is therefore not strictly justified;
this does not lead to bias in point estimates, but the Hessian-based
approximation is likely to underestimate the true variance somewhat.
Accordingly, we have made some simple adjustments to ``effective
sample size'' based on summaries of the simulated datasets. This
should be quite adequate for design purposes— since, in any case,
all our variance estimates have to be based on uncertain assumptions
about true parameter values— but a more detailed treatment may be
worthwhile when it comes to analysing the real data.

A general and comprehensive treatment of non-independence in CKMR
is beyond the scope of this paper. We restrict attention to some obvious
aspects for walrus that are easy to address. We consider the comparisons
in stages: SPs, then MOPs, then XmHSPs. We adjust set the effective
sample size for each stage based on recaptures from the preceding
stages in one simulated dataset, as follows:
\begin{itemize}
\item Sample sizes are initially taken from the simulated dataset (thus
allowing detailed breakdown of sample size by age, year, etc). All
available samples are used for SP comparisons.
\item If an individual is self-recaptured, only its final capture will be
used in MOP and XmHSP comparisons (i.e. duly adjusting the sample
sizes sample sizes for MOPs and XmHSPs, as well as the number of MOPs
etc found if that individual is involved).
\item Any Offspring $o$ identified in a MOP, will be excluded from XmHSP
comparisons (since $o$'s sibship with any other sample $i$ can be
deduced from the MOP results, based on whether $i$ is also an offspring
of $o$'s Mother).
\end{itemize}
This deals with the implications of one type of kinship for the others,
but does not deal with multiple recaptures within a kinship class
(e.g. an individual who is sampled 3 times; given that sample A matches
sample B, and B matches C, it is redundant to compare A with C). There
are simple ways to handle that with real datasets, as long as age
is known fairly accurately.

\pagebreak{}

\section{Model checking\label{sec:Model-checking}}

Close-kin pairwise probability formulae are usually quite simple,
at least with hindsight, but they still can be awkard to get right
in the first place. One way to reduce the risk of mistakes is to generate
simulated datasets, and check that the CKMR code is giving the expected
results when known parameter values are inserted. CKMR simulation
code looks utterly different from kinship-probability code, and the
chance of ``making the same mistake twice'' is therefore much less
than with many statistical simulations. Robustness is improved even
further if two different people are involved, one to simulate and
one to write kinship-probability code. Even though simulation is not
strictly necessary for most CKMR design exercises, simulation may
be worth the additional effort in order to help the whole process,
and that is the approach we took for walrus. We did find and fix several
mistakes this way, both in the CKMR code and in the simulation code,
so the exercise was certainly worthwhile.

The obvious question is how to approach CKMR model-checking when simulated
datasets are available. There are various options and no . One thing
to avoid, if possible, is the naive and laborious approach of actually
\emph{fitting} a CKMR to each simulated dataset, which can be painfully
slow. %
\begin{lyxgreyedout}
(Note, perhaps for discussion: We started this project before RTMB
became available, expecting that the actual model-fitting code for
real data would eventually have to be written in TMB itself, but keen
to avoid the complexity of TMB at the design stage. In contrast, design
calculations are quick because it is only necessary to calculate probability
arrays once, and R alone is adequately fast, without TMB or RTMB.
However, it would not be practical to fit even our simple model to
multiple datasets without RTMB; and even with RTMB, repeated fitting
of a more complicted model, e.g. with copious random effects, might
be a challenge.)%
\end{lyxgreyedout}
{} We used several checks. All are aimed at detecting gross errors (and
we did find some); power to detect subtle mistakes is lower, but in
our experience subtle mistakes are actually less likely than big ones.
The first two checks are based on single realizations of simulated
data, and so are also suitable as diagnostics when fitting to real
data; the last two require multiple simulated datasets.

\begin{comment}
One option is to include final result(s) for each check right here,
after describing the check. Otherwise, if this section goes mainly
into Methods, the results of each little check will be a long way
from the text describing it, and the reader will have forgotten what
the check is by the time they see its results. I absolutely hate papers
like that :) But, it might depend; needs Zoom.
\end{comment}

\begin{itemize}
\item Observed and expected totals of sampled kin-pairs of each type. Clearly,
unless these match reasonably well, there must be a major inconsistency
between model and simulationg. The definition of ``reasonably well''
can be guided by the inherent Poisson variability. If an expected
total is 227, say, then we would not expect to see observed total
much outside, say, the 95\% confidence limits for a Poisson distribution
with mean (and therefore variance) 227. This can be roughly approximated
by $227\pm2\sqrt{227}$ or about {[}195,255{]}. Clearly, the expected
total needs to be fairly large for this to have much power, so it
might be useful to increase the simulated sample size for checking
purposes.\\
{*}{*}OPTION{*}{*}list the totals here (for first test dataset, chosen
so that sim matches CK code as closely as possible)
\item Breakdown of observed and expected kin-pair totals across some covariate
of interest. If the totals from the previous step are not matching
well, then the breakdown may shed light on where to look for problems.
For example: the distribution of birth-gaps between XmHSPs is driven
in the longer term by the adult rate mortality rate, so if observed
and expected do not correspond, then the treatment of mortality is
likely inconsistent. Also, the number of mothers by age-at-birth should
fluctuate over the first few years of adulthood because of the typically-three-year
breeding cycle (most 6yo have just given birth; most 7yo are still
nursing last year's offspring, etc), until it settles down because
of the averaging effects of irregularities. If the observed and expected
patterns do not match, then the breeding cycle treatment is inconsistent.\\
{*}{*}OPTION{*}{*} show the 2 graphs here.
\item P-values of observed kin-totals by type, based on the Poisson distribution
as above. Given a reasonable number of simulated datasets (say 20
or more), these should be roughly uniform across the interval {[}0,1{]}.
Clearly, it would require a large number of simulations to get a precise
check here, but precision is not necessary: the goal is to pick up
fairly coarse errors.\\
{*}{*}OPTION{*}{*} show 4 histos here (instead of box'n'whiska)
\item Looking at the mean and variance of the derivative of the pseudo-log-likelihood
at the true parameter values $\theta_{0}$ (something which can be
calculated fairly quickly by numerical differentiation). The mean
should be close to 0 and the variance determines what ``close''
might mean, given the number of simulations available. This checks
the crucial ``unbiased estimating equation'' (UEE) assumption required
by most statistical estimation frameworks, including maximum-likelihood.
If UEE does not hold, then by definition there is a mismatch between
simulation and model.\\
{*}{*}OPTION{*}{*} there's some numbers printeed at the end of compare2sims.R,
I thnk.
\end{itemize}
The description so far implicitly assumes that the CKMR model (if
working right) corresponds exactly to the data-generation mechanism
in the simulations. However, it might be desirable to make the CKMR
model simpler, especially for design purposes where the goal is just
to make sure that sampling plans are sensible; developing a more complicated
and realistic model can often be left until the real data appears.
For example, we wanted to avoid reproductive senescence in the CKMR
equations, so that all adults could be treated as a single block without
requiring age-structured dynamics inside the model. Nevertheless,
senescence is likely a reality of the walrus world, and there is such
a thing as ``too simple to be useful'', so it is worth checking
whether the simpler formulation is going to run into serious trouble.
Simulated datasets can be used to estimate approximate bias in a slightly-mis-specified
CKMR model, again without needing to do any estimation. The idea is
to approximate the MLE for each dataset, based only on calculations
using the true parameter value for the simulations. The MLE $\hat{\theta}$
will by definition satisfy the equation $\left.d\Lambda/d\theta\right\vert _{\hat{\theta}}=0$,
and we can take a first-order Taylor expansion around the true value
$\theta_{0}$ to give

\begin{gather}
0=\left.\frac{d\Lambda}{d\theta}\right\vert _{\hat{\theta}}\approx\left.\frac{d\Lambda}{d\theta}\right\vert _{\theta_{0}}+\left(\hat{\theta}-\theta_{0}\right)\left.\frac{d^{2}\Lambda}{d\theta^{2}}\right\vert _{\theta_{0}}\nonumber \\
\implies\hat{\theta}-\theta_{0}\approx-\left[\left.\frac{d\Lambda^{2}}{d\theta^{2}}\right\vert _{\theta_{0}}\right]^{-1}\left.\frac{d\Lambda}{d\theta}\right\vert _{\theta_{0}}\label{eq:bias-approx}
\end{gather}

The square-bracketed term can be replaced (to the same order of accuracy
as the rest of the approxmation) by the \emph{expected} Hessian which
is the crux of our design calculations anyway, and which of course
does not vary from one simulation to the next. Thus, the only quantity
that has to be calculated per simulated dataset is $\left.d\Lambda/d\theta\right\vert _{\theta_{0}}$,
already required for the unbiased-estimating-equation check above.
The estimated bias is the average across simulations of (\ref{eq:bias-approx}).
This is quite similar to the UEE check above, but with a change in
focus: this time, we may be prepared to tolerate some small violation
of UEE, provided that it does not imply substantial bias on the parameter
scale. In particular, if the estimated bias for the $r$\textsuperscript{th}
parameter (i.e. $r$\textsuperscript{th } component of $\theta$)
is below its sampling variability— say, if bias is less than 1 standard
deviation, computed from the square-root of the diagonal of the inverse
Hessian or $\sqrt{H^{-1}\left(r,r\right)}$— then there is little
reason to worry about bias for that particular parameter. 

{*}{*}OPTION{*}{*} stuff from the end of compare2sims.R

{*}{*}DISCUSSION?{*}{*}

In the end, based on the checks above, our estimation and simulation
codes did indeed appear consistent, and any bias induced by (among
other minor things) ignoring senescence did not seem problematic.
Of course, we only reached that position \emph{after} going thru the
checking process several times, to find and fix inconsistencies.


\pagebreak{}

\section{Additional results\label{sec:Additional-results}}

\selectlanguage{american}%
\begin{table}
\caption{Expected CVs on adult female survival, juvenile female survival, and
the proportion of adult females in breeding state 2 under different
demographic and sampling scenarios with and without the use of lethal
samples and CKMR.\label{tab:LH_Expected-CVs}}

\begin{tabular}{|c|c|c|c|c|c|c|}
\hline 
\begin{cellvarwidth}[t]
\centering
Demographic 

Scenario
\end{cellvarwidth} & \begin{cellvarwidth}[t]
\centering
Lethal 

Samples
\end{cellvarwidth} & \begin{cellvarwidth}[t]
\centering
Sampling 

Scenario
\end{cellvarwidth} & CKMR & \begin{cellvarwidth}[t]
\centering
Adult Female 

Survival
\end{cellvarwidth} & \begin{cellvarwidth}[t]
\centering
Juvenile Female 

Survival
\end{cellvarwidth} & \begin{cellvarwidth}[t]
\centering
P. Adult Female 

in State 2
\end{cellvarwidth}\tabularnewline
\hline 
\hline 
1 & No & 1 & Yes & 0.01 & 0.05 & 0.1\tabularnewline
\hline 
1 & No & 2 & Yes & 0.01 & 0.04 & 0.09\tabularnewline
\hline 
1 & No & 3 & Yes & 0.01 & 0.04 & 0.09\tabularnewline
\hline 
1 & No & 4 & Yes & 0.02 & 0.05 & 0.11\tabularnewline
\hline 
1 & No & 5 & Yes & 0.01 & 0.05 & 0.1\tabularnewline
\hline 
1 & No & 6 & Yes & 0.01 & 0.05 & 0.1\tabularnewline
\hline 
1 & No & 7 & Yes & 0.01 & 0.05 & 0.1\tabularnewline
\hline 
1 & Yes & 1 & Yes & 0.01 & 0.04 & 0.1\tabularnewline
\hline 
1 & Yes & 2 & Yes & 0.01 & 0.04 & 0.09\tabularnewline
\hline 
1 & Yes & 3 & Yes & 0.01 & 0.04 & 0.09\tabularnewline
\hline 
1 & Yes & 4 & Yes & 0.01 & 0.05 & 0.1\tabularnewline
\hline 
1 & Yes & 5 & Yes & 0.01 & 0.04 & 0.1\tabularnewline
\hline 
1 & Yes & 6 & Yes & 0.01 & 0.04 & 0.1\tabularnewline
\hline 
1 & Yes & 7 & Yes & 0.01 & 0.05 & 0.09\tabularnewline
\hline 
2 & No & 1 & Yes & 0.01 & 0.03 & 0.06\tabularnewline
\hline 
2 & No & 2 & Yes & 0.01 & 0.02 & 0.05\tabularnewline
\hline 
2 & No & 3 & Yes & 0.01 & 0.02 & 0.05\tabularnewline
\hline 
2 & No & 4 & Yes & 0.01 & 0.03 & 0.07\tabularnewline
\hline 
2 & No & 5 & Yes & 0.01 & 0.03 & 0.06\tabularnewline
\hline 
2 & No & 6 & Yes & 0.01 & 0.03 & 0.06\tabularnewline
\hline 
2 & No & 7 & Yes & 0.01 & 0.03 & 0.06\tabularnewline
\hline 
2 & Yes & 1 & Yes & 0.01 & 0.03 & 0.06\tabularnewline
\hline 
2 & Yes & 2 & Yes & 0.01 & 0.02 & 0.05\tabularnewline
\hline 
2 & Yes & 3 & Yes & 0.01 & 0.02 & 0.05\tabularnewline
\hline 
2 & Yes & 4 & Yes & 0.01 & 0.03 & 0.06\tabularnewline
\hline 
2 & Yes & 5 & Yes & 0.01 & 0.03 & 0.06\tabularnewline
\hline 
2 & Yes & 6 & Yes & 0.01 & 0.03 & 0.06\tabularnewline
\hline 
2 & Yes & 7 & Yes & 0.01 & 0.03 & 0.06\tabularnewline
\hline 
3 & No & 1 & Yes & 0.02 & 0.06 & 0.13\tabularnewline
\hline 
3 & No & 2 & Yes & 0.02 & 0.05 & 0.11\tabularnewline
\hline 
3 & No & 3 & Yes & 0.02 & 0.05 & 0.11\tabularnewline
\hline 
3 & No & 4 & Yes & 0.02 & 0.07 & 0.14\tabularnewline
\hline 
3 & No & 5 & Yes & 0.02 & 0.06 & 0.13\tabularnewline
\hline 
3 & No & 6 & Yes & 0.02 & 0.06 & 0.13\tabularnewline
\hline 
3 & No & 7 & Yes & 0.02 & 0.06 & 0.12\tabularnewline
\hline 
3 & Yes & 1 & Yes & 0.02 & 0.06 & 0.12\tabularnewline
\hline 
3 & Yes & 2 & Yes & 0.01 & 0.05 & 0.11\tabularnewline
\hline 
3 & Yes & 3 & Yes & 0.01 & 0.05 & 0.11\tabularnewline
\hline 
3 & Yes & 4 & Yes & 0.02 & 0.06 & 0.13\tabularnewline
\hline 
3 & Yes & 5 & Yes & 0.02 & 0.06 & 0.12\tabularnewline
\hline 
3 & Yes & 6 & Yes & 0.02 & 0.06 & 0.12\tabularnewline
\hline 
3 & Yes & 7 & Yes & 0.02 & 0.06 & 0.12\tabularnewline
\hline 
1 & No & 1 & No & 0.03 & 0.07 & 1.01\tabularnewline
\hline 
1 & No & 2 & No & 0.03 & 0.06 & 0.92\tabularnewline
\hline 
1 & No & 3 & No & 0.03 & 0.06 & 0.92\tabularnewline
\hline 
1 & No & 4 & No & 0.04 & 0.08 & 1.1\tabularnewline
\hline 
1 & No & 5 & No & 0.04 & 0.08 & 1.04\tabularnewline
\hline 
1 & No & 6 & No & 0.04 & 0.08 & 1.04\tabularnewline
\hline 
1 & No & 7 & No & 0.03 & 0.07 & 1\tabularnewline
\hline 
1 & Yes & 1 & No & 0.03 & 0.06 & 0.94\tabularnewline
\hline 
1 & Yes & 2 & No & 0.02 & 0.06 & 0.86\tabularnewline
\hline 
1 & Yes & 3 & No & 0.02 & 0.06 & 0.86\tabularnewline
\hline 
1 & Yes & 4 & No & 0.04 & 0.07 & 1.03\tabularnewline
\hline 
1 & Yes & 5 & No & 0.03 & 0.07 & 0.97\tabularnewline
\hline 
1 & Yes & 6 & No & 0.03 & 0.07 & 0.97\tabularnewline
\hline 
1 & Yes & 7 & No & 0.03 & 0.06 & 0.94\tabularnewline
\hline 
2 & No & 1 & No & 0.02 & 0.04 & 0.6\tabularnewline
\hline 
2 & No & 2 & No & 0.01 & 0.03 & 0.54\tabularnewline
\hline 
2 & No & 3 & No & 0.01 & 0.03 & 0.54\tabularnewline
\hline 
2 & No & 4 & No & 0.02 & 0.05 & 0.65\tabularnewline
\hline 
2 & No & 5 & No & 0.02 & 0.04 & 0.61\tabularnewline
\hline 
2 & No & 6 & No & 0.02 & 0.04 & 0.61\tabularnewline
\hline 
2 & No & 7 & No & 0.02 & 0.04 & 0.59\tabularnewline
\hline 
2 & Yes & 1 & No & 0.01 & 0.04 & 0.55\tabularnewline
\hline 
2 & Yes & 2 & No & 0.01 & 0.03 & 0.51\tabularnewline
\hline 
2 & Yes & 3 & No & 0.01 & 0.03 & 0.51\tabularnewline
\hline 
2 & Yes & 4 & No & 0.02 & 0.04 & 0.6\tabularnewline
\hline 
2 & Yes & 5 & No & 0.01 & 0.04 & 0.56\tabularnewline
\hline 
2 & Yes & 6 & No & 0.01 & 0.04 & 0.56\tabularnewline
\hline 
2 & Yes & 7 & No & 0.01 & 0.04 & 0.55\tabularnewline
\hline 
3 & No & 1 & No & 0.04 & 0.09 & 1.25\tabularnewline
\hline 
3 & No & 2 & No & 0.03 & 0.08 & 1.13\tabularnewline
\hline 
3 & No & 3 & No & 0.03 & 0.08 & 1.13\tabularnewline
\hline 
3 & No & 4 & No & 0.06 & 0.11 & 1.36\tabularnewline
\hline 
3 & No & 5 & No & 0.05 & 0.1 & 1.29\tabularnewline
\hline 
3 & No & 6 & No & 0.05 & 0.1 & 1.29\tabularnewline
\hline 
3 & No & 7 & No & 0.04 & 0.09 & 1.24\tabularnewline
\hline 
3 & Yes & 1 & No & 0.04 & 0.08 & 1.17\tabularnewline
\hline 
3 & Yes & 2 & No & 0.03 & 0.07 & 1.07\tabularnewline
\hline 
3 & Yes & 3 & No & 0.03 & 0.07 & 1.07\tabularnewline
\hline 
3 & Yes & 4 & No & 0.05 & 0.09 & 1.27\tabularnewline
\hline 
3 & Yes & 5 & No & 0.04 & 0.09 & 1.21\tabularnewline
\hline 
3 & Yes & 6 & No & 0.04 & 0.09 & 1.21\tabularnewline
\hline 
3 & Yes & 7 & No & 0.04 & 0.08 & 1.16\tabularnewline
\hline 
\end{tabular}
\end{table}

\begin{table}
\selectlanguage{english}%
\caption{Expected CV on adult female population size in 2015, 2020, and 2025
with different demographic and sampling scenarios and with and without
the use of lethal samples and CKMR. \label{tab:N_Expected-CV}}

\begin{tabular}{|c|c|c|c|c|c|c|}
\hline 
\begin{cellvarwidth}[t]
\centering
Demographic 

Scenario
\end{cellvarwidth} & \begin{cellvarwidth}[t]
\centering
Lethal 

Samples
\end{cellvarwidth} & \begin{cellvarwidth}[t]
\centering
Sampling 

Scenario
\end{cellvarwidth} & CKMR & \begin{cellvarwidth}[t]
\centering
2015 

Adult Females
\end{cellvarwidth} & \begin{cellvarwidth}[t]
\centering
2020 

Adult Females
\end{cellvarwidth} & \begin{cellvarwidth}[t]
\centering
2025 

Adult Females
\end{cellvarwidth}\tabularnewline
\hline 
\hline 
1 & No & 1 & Yes & 0.07 & 0.1 & 0.14\tabularnewline
\hline 
1 & No & 2 & Yes & 0.06 & 0.08 & 0.12\tabularnewline
\hline 
1 & No & 3 & Yes & 0.06 & 0.08 & 0.12\tabularnewline
\hline 
1 & No & 4 & Yes & 0.08 & 0.12 & 0.16\tabularnewline
\hline 
1 & No & 5 & Yes & 0.07 & 0.1 & 0.14\tabularnewline
\hline 
1 & No & 6 & Yes & 0.07 & 0.1 & 0.14\tabularnewline
\hline 
1 & No & 7 & Yes & 0.07 & 0.1 & 0.14\tabularnewline
\hline 
1 & Yes & 1 & Yes & 0.07 & 0.09 & 0.13\tabularnewline
\hline 
1 & Yes & 2 & Yes & 0.06 & 0.08 & 0.11\tabularnewline
\hline 
1 & Yes & 3 & Yes & 0.06 & 0.08 & 0.11\tabularnewline
\hline 
1 & Yes & 4 & Yes & 0.07 & 0.11 & 0.15\tabularnewline
\hline 
1 & Yes & 5 & Yes & 0.07 & 0.09 & 0.13\tabularnewline
\hline 
1 & Yes & 6 & Yes & 0.07 & 0.09 & 0.13\tabularnewline
\hline 
1 & Yes & 7 & Yes & 0.07 & 0.09 & 0.13\tabularnewline
\hline 
2 & No & 1 & Yes & 0.04 & 0.05 & 0.08\tabularnewline
\hline 
2 & No & 2 & Yes & 0.03 & 0.04 & 0.06\tabularnewline
\hline 
2 & No & 3 & Yes & 0.03 & 0.04 & 0.06\tabularnewline
\hline 
2 & No & 4 & Yes & 0.04 & 0.06 & 0.09\tabularnewline
\hline 
2 & No & 5 & Yes & 0.04 & 0.05 & 0.08\tabularnewline
\hline 
2 & No & 6 & Yes & 0.04 & 0.05 & 0.08\tabularnewline
\hline 
2 & No & 7 & Yes & 0.04 & 0.05 & 0.08\tabularnewline
\hline 
2 & Yes & 1 & Yes & 0.03 & 0.05 & 0.07\tabularnewline
\hline 
2 & Yes & 2 & Yes & 0.03 & 0.04 & 0.06\tabularnewline
\hline 
2 & Yes & 3 & Yes & 0.03 & 0.04 & 0.06\tabularnewline
\hline 
2 & Yes & 4 & Yes & 0.04 & 0.05 & 0.08\tabularnewline
\hline 
2 & Yes & 5 & Yes & 0.03 & 0.05 & 0.07\tabularnewline
\hline 
2 & Yes & 6 & Yes & 0.03 & 0.05 & 0.07\tabularnewline
\hline 
2 & Yes & 7 & Yes & 0.03 & 0.05 & 0.07\tabularnewline
\hline 
3 & No & 1 & Yes & 0.09 & 0.13 & 0.18\tabularnewline
\hline 
3 & No & 2 & Yes & 0.08 & 0.11 & 0.15\tabularnewline
\hline 
3 & No & 3 & Yes & 0.08 & 0.11 & 0.15\tabularnewline
\hline 
3 & No & 4 & Yes & 0.1 & 0.15 & 0.21\tabularnewline
\hline 
3 & No & 5 & Yes & 0.09 & 0.13 & 0.18\tabularnewline
\hline 
3 & No & 6 & Yes & 0.09 & 0.13 & 0.18\tabularnewline
\hline 
3 & No & 7 & Yes & 0.09 & 0.13 & 0.18\tabularnewline
\hline 
3 & Yes & 1 & Yes & 0.08 & 0.12 & 0.17\tabularnewline
\hline 
3 & Yes & 2 & Yes & 0.08 & 0.1 & 0.14\tabularnewline
\hline 
3 & Yes & 3 & Yes & 0.08 & 0.1 & 0.14\tabularnewline
\hline 
3 & Yes & 4 & Yes & 0.09 & 0.14 & 0.19\tabularnewline
\hline 
3 & Yes & 5 & Yes & 0.09 & 0.12 & 0.17\tabularnewline
\hline 
3 & Yes & 6 & Yes & 0.09 & 0.12 & 0.17\tabularnewline
\hline 
3 & Yes & 7 & Yes & 0.08 & 0.12 & 0.17\tabularnewline
\hline 
1 & No & 1 & No & 0.14 & 0.19 & 0.31\tabularnewline
\hline 
1 & No & 2 & No & 0.14 & 0.15 & 0.23\tabularnewline
\hline 
1 & No & 3 & No & 0.14 & 0.15 & 0.23\tabularnewline
\hline 
1 & No & 4 & No & 0.15 & 0.25 & 0.42\tabularnewline
\hline 
1 & No & 5 & No & 0.15 & 0.21 & 0.34\tabularnewline
\hline 
1 & No & 6 & No & 0.15 & 0.21 & 0.34\tabularnewline
\hline 
1 & No & 7 & No & 0.14 & 0.19 & 0.31\tabularnewline
\hline 
1 & Yes & 1 & No & 0.14 & 0.17 & 0.27\tabularnewline
\hline 
1 & Yes & 2 & No & 0.13 & 0.14 & 0.21\tabularnewline
\hline 
1 & Yes & 3 & No & 0.13 & 0.14 & 0.21\tabularnewline
\hline 
1 & Yes & 4 & No & 0.14 & 0.21 & 0.36\tabularnewline
\hline 
1 & Yes & 5 & No & 0.14 & 0.19 & 0.3\tabularnewline
\hline 
1 & Yes & 6 & No & 0.14 & 0.19 & 0.3\tabularnewline
\hline 
1 & Yes & 7 & No & 0.13 & 0.17 & 0.27\tabularnewline
\hline 
2 & No & 1 & No & 0.06 & 0.09 & 0.15\tabularnewline
\hline 
2 & No & 2 & No & 0.06 & 0.07 & 0.11\tabularnewline
\hline 
2 & No & 3 & No & 0.06 & 0.07 & 0.11\tabularnewline
\hline 
2 & No & 4 & No & 0.07 & 0.12 & 0.2\tabularnewline
\hline 
2 & No & 5 & No & 0.07 & 0.1 & 0.16\tabularnewline
\hline 
2 & No & 6 & No & 0.07 & 0.1 & 0.16\tabularnewline
\hline 
2 & No & 7 & No & 0.06 & 0.09 & 0.15\tabularnewline
\hline 
2 & Yes & 1 & No & 0.06 & 0.08 & 0.13\tabularnewline
\hline 
2 & Yes & 2 & No & 0.06 & 0.07 & 0.1\tabularnewline
\hline 
2 & Yes & 3 & No & 0.06 & 0.07 & 0.1\tabularnewline
\hline 
2 & Yes & 4 & No & 0.06 & 0.1 & 0.17\tabularnewline
\hline 
2 & Yes & 5 & No & 0.06 & 0.09 & 0.14\tabularnewline
\hline 
2 & Yes & 6 & No & 0.06 & 0.09 & 0.14\tabularnewline
\hline 
2 & Yes & 7 & No & 0.06 & 0.08 & 0.13\tabularnewline
\hline 
3 & No & 1 & No & 0.19 & 0.25 & 0.41\tabularnewline
\hline 
3 & No & 2 & No & 0.18 & 0.2 & 0.31\tabularnewline
\hline 
3 & No & 3 & No & 0.18 & 0.2 & 0.31\tabularnewline
\hline 
3 & No & 4 & No & 0.2 & 0.33 & 0.56\tabularnewline
\hline 
3 & No & 5 & No & 0.19 & 0.28 & 0.45\tabularnewline
\hline 
3 & No & 6 & No & 0.19 & 0.28 & 0.45\tabularnewline
\hline 
3 & No & 7 & No & 0.19 & 0.25 & 0.42\tabularnewline
\hline 
3 & Yes & 1 & No & 0.18 & 0.23 & 0.36\tabularnewline
\hline 
3 & Yes & 2 & No & 0.18 & 0.19 & 0.28\tabularnewline
\hline 
3 & Yes & 3 & No & 0.18 & 0.19 & 0.28\tabularnewline
\hline 
3 & Yes & 4 & No & 0.19 & 0.28 & 0.47\tabularnewline
\hline 
3 & Yes & 5 & No & 0.19 & 0.25 & 0.41\tabularnewline
\hline 
3 & Yes & 6 & No & 0.19 & 0.25 & 0.41\tabularnewline
\hline 
3 & Yes & 7 & No & 0.18 & 0.22 & 0.36\tabularnewline
\hline 
\end{tabular}\selectlanguage{english}%
\end{table}
\selectlanguage{english}%

