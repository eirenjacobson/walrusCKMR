\selectlanguage{british}%

\begin{itemize}
\item We show how sample collection plans could be modified to achieve desired
monitoring goals with less sampling effort.
\item We didn't bother doing X coz IJAD\footnote{It's Just A Design}. For
real data analysis, we might do Y instead.
\item Ways to extend the model... impact of DNAge
\item Something about adult males and paternal half-siblings (sec 2.1 refers
to discussion)
\item Future utility of lethal samples (although my guess is: there won't
be enough. Glass-half-full, or glass-half-empty, if you're a walrus?)
\item The full ramifications of opting for a stage-structured quasi-equilibrium
model, which avoids having to model age composition but does entail
an \emph{assumption} about selectivity, are not at all obvious, but
the model seems to us fairly reasonable; it might be worth revisiting
when large numbers of DNAge samples become available. At that point
it would be possible to compare the actual age compositions with the
predicted compositions assuming partly-unselective sampling and quasi-equilibrium.
\item on stage-structured dynamics: That assumption may turn out to be unreasonable
for juveniles especially; but it will only be possible to check once
enough sample-age-composition data become available. However, if it
does turn out to be the case that (say) 2yo are disproportionately
likely to be sampled (given their estimated abundance from the fitted
model), then it would not be hard to adjust the stage-structured IMR
equations to incorporate sample-composition-data and (estimated) selectivity.
Sample sizes in this project are large enough that selectivity (i.e.,
the ratio of age-specific sample compositions to model-estimated population
age compositions) should be estimated with respectable precision and
without \textquotedbl propagating\textquotedbl{} a lot of uncertainty
into other parameter estimates. We therefore think that our current
somewhat crude IMR sub-model should give a reasonable guide to ultimate
precision, even if it gets adjusted somewhat in the cold light of
real data. Note that similar assumptions appear to be made in Beatty
et al. 202 (to be confirmed).
\selectlanguage{english}%
\item A purely-age-structured version of \eqref{eq:self-staged} would need
to explicitly keep track of numbers-at-age, not just adult abundance
(as would the other kin types). The quasi-equilibrium assumption might
allow us to do this, but that assumption directly constrains relative
abundances-at-age. In practice, a fully age-structured CKMR formulation
for walrus will need something more sophisticated and time-varying
than a quasi-equilibrium age distribution, and therefore additional
parameters to estimate. We therefore opted for a stage- rather than
age-structured SP model in the hope that the overall statistical information
content about total abundance is reasonably realistic compared to
what we might get from a more complicated population dynamics model.
\item While a stable-age-composition between 2000–2027 is probably not valid
for the entire range of adult ages—since older adults would have experienced
long periods of increased mortality from hunting—it is perhaps a reasonable
assumption for younger adults, and it is only younger adults that
matter here because they indirectly determine the number of juveniles.
A stable age composition for juveniles seems fairly reasonable, since
\textquotedbl recruitment variability\textquotedbl{} cannot be high
for an animal with a litter size of 1, and it only requires a few
years for the juvenile distribution to settle down.
\selectlanguage{british}%
\item As should be evident from the preceding text and number of authors
on this paper, building a close-kin model involves a high level of
collaboration between statisticians, biologists and geneticists. CKMR
is very much a multidisciplinary methodology and each discipline has
a great deal to input into the process of model building.
\item Could mention that CKMR was motivated by fisheries and is an example
of a shared tool between fisheries scientists and ecologists, maybe
cite Schaub et al 2024
\item appendix \eqref{sec:Skip-breeding} discusses skip-breeding\selectlanguage{english}%
\end{itemize}

